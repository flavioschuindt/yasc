\section{Introdução\label{sec:Introducao}}

	\indent\indent Este relatório destina-se a cobrir a implementação em \emph{C} de uma calculadora em rede numa estrutura de cliente - servidor, implementada segundo as normas ANSI C99 e POSIX.

	Na nossa abordagem, optámos por uma implementação monolítica de \emph{IO} síncrono para o cliente, e uma estrutura altamente paralelizada e completamente assíncrona para o servidor.
	Estas opções serão discutidas ao longo deste relatório, assim como algumas funcionalidades previstas, mas que não chegámos a implementar.

	A estrutura deste documento corresponderá a uma descrição tanto do cliente como do servidor, seguidos de uma análise crítica ao seu desempenho na forma final, submetida juntamente com este relatório.

	Neste relatório, incluiem-se ainda em anexo um diagrama do conceito abordado (\ref{app:diag}), assim como os manuais tanto do cliente (\ref{app:manC}) como do servidor (\ref{app:manS}).
	Contudo uma explicação sumária do seu funcionamente é dada também no corpo do documento.

	O desenvolvimento deste projecto foi constantemente registado num repositório \emph{git} online que pode ser consultado em \href{https://bitbucket.org/BrunoMSantos/yasc/overview}{\emph{bitbucket.org}}.

\clearpage