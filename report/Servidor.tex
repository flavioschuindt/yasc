\section{Servidor\label{sec:Servidor}}


\subsection{Chamada\label{sec:servidor_chamada}}

	\indent\indent A invocação do servidor segue estritamente os requisitos fornecidos pelo enunciado do projecto, bastando indicar como argumento a porta de escuta na qual o servidor deverá esperar clientes.

	\begin{verbatim}
		yascS.exe <port>
	\end{verbatim}

	O utilizador pode ainda escolher se deseja que o modo verboso seja activado no servidor passando a \emph{flag} \verb|-v| na invocação do mesmo.
	O modo verboso permite que seja mostrado na saída padrão do servidor informações relativas à entrada e saída de clientes e ao lançamento de novos \emph{workers} no \emph{pool} de \emph{threads}.
	A activação ou desactivação do modo verboso pode ser feita a qualquer momento pelo administrador do servidor, não sendo vinculativo o uso da \emph{flag}.

	A execução do servidor num directório que não contenha \verb|/doc| não permite o acesso ao ficheiro de ajuda durante a execução.

\clearpage
\subsection{Interface\label{sec:servidor_interface}}

	\indent\indent O servidor disponibiliza uma interface para o administrador do mesmo e esta é feita à custa de comandos:

	\begin{itemize}
		\setlength{\parskip}{-3pt}
		\item{Gestão do servidor:}\begin{itemize}
			\setlength{\parskip}{-3pt}
			\item{M}
		\end{itemize}
		\item{Outros:}\begin{itemize}
			\setlength{\parskip}{-3pt}
			\item{V}
			\item{HELP}
			\item{F}
		\end{itemize}
	\end{itemize}

	O comando \verb|M| é o único comando de gestão disponível.
	Este permite "tirar uma foto" do instante actual do servidor, isto é, quais clientes estão conectados e como está a pilha de cada cliente individualmente.
	O resultado é exibido na seguinte forma:

	\begin{verbatim}
		IP                [stack]
		xxx.xxx.xxx.xxx   [stack top, ..., stack bottom]
	\end{verbatim}

	Em que o campo IP é um endereço IPv4, e os valores de \emph{stack} são inteiros de 32 bits com sinal compreendidos entre \verb|INT_MIN| e \verb|INT_MAX|, formatados em base decimal.

	De notar que o comando \verb|M| é um comando dispendioso em termos de performance, como será visto na secção \ref{sec:servidor_estrutura}.
	O seu uso constante bloqueia o tratamento dos clientes.

	O comando V permite ligar ou desligar o modo verboso.
	Correntemente, este modo permite ter melhor noção do que está a acontecer no servidor em tempo real, mas só no que diz respeito à entrada e saída de clientes, e ao lançamento de \emph{threads}.

	Os comandos restantes destinam-se a abrir o manual do servidor e a fechar o servidor.
	Tal como no caso do cliente, é necessário que o manual esteja ao alcance do programa conforme discutido na secção \ref{sec:servidor_chamada}.

	Além das formas de \emph{output} já abordadas, o servidor no ecrã resultados de erros internos, que invariavelmente levam à terminação do programa.


\clearpage
\subsection{Estrutura lógica\label{sec:servidor_estrutura}}

	\indent\indent O esqueleto do servidor é constituído por três unidades computacionais distintas que controlam todos os serviços prestados.
	São elas o servidor Master, o \emph{Manager} e o \emph{pool} \emph{Manager}.

	Vale referir que não lançamos novos processos, por serem mais pesados e não oferecerem nenhuma vantagem em relação às \emph{threads} no contexto do nosso projecto.
	Isto é válido desde que, claro está, os problemas de sincronização sejam tidos em conta.

	\emph{Threads}, por outro lado, lançamos muitas além das três referidas.
	O \emph{pool} \emph{Manager} encarrega-se de criar ou matar \emph{threads} com o único propósito de atender aos clientes -- \emph{threads} chamadas de \emph{workers} ou \emph{slaves} a partir deste ponto.


	A razão por detrás desta abordagem foi um conceito de design que definimos bem cedo: queríamos uma \emph{pool} de \emph{threads} em que cada \emph{worker} fosse capaz de atender a vários clientes.
	Como os clientes nem sempre estão a executar comandos, não só não faz sentido ter unidades computacionais bloqueadas no \verb|read()|, como não precisamos de tantas \emph{threads}.
	Daqui vem uma grande vantagem na nossa implementação que é o facto de que podermos ter muitos mais clientes do que \emph{threads}.

	Contudo esta implementação sacrifica alguma simplicidade na sincronização dos vários fios de execução e, sobretudo, a conveniência de bloquear automaticamente na chamada ao \verb|read()|.

	\setlength{\parskip}{20pt}		% distance between paragraphs

	Definindo as funções de cada fio de execução, o \emph{Manager} tem a única função de atender aos comandos do administrador, partilhando algum do código da função de \emph{parsing} do cliente.
	Executa maioritariamente de forma completamente paralela, excepto quando o administrador pede a listagem de clientes com sessão iniciada.
	Aí, as condições de corrida são importantes e levou-nos ao uso de um \emph{mutex} para controlar o acesso a esses recursos.

	\setlength{\parskip}{5pt}		% distance between paragraphs

	Analisando o código, a região crítica na função \verb|print_client_info()| é na verdade um pouco inferior à região protegida com a tranca.
	Tal deve-se a uma tentativa de minimizar as hipóteses de uma outra \emph{thread} interromper a listagem, permitindo uma melhor visualização da lista.
	Dito isto, o comando \verb|M| destina-se a efeitos de debugging e administração, pelo que o seu peso deverá ser evitado em condições de funcionamento normal.

	\setlength{\parskip}{20pt}		% distance between paragraphs

	O módulo Master fica bloqueado grande parte do seu tempo de execução posterior ao setup do servidor.
	Ao aceitar uma nova conexão de um cliente, o Master server invoca a função \verb|add_client()|, colocando-o na lista de clientes com sessão aberta.
	É importante ressaltar que o número máximo de clientes a serem aceites está limitado estaticamente por \verb|MAX_CLIENTS|.
	Se o número de clientes na lista atingir esse máximo, então o servidor mestre pára de aceitar novas conexões e fica em \emph{idle}, dormindo por um tempo definido em \verb|DOORMAN_DOZE|.

	\setlength{\parskip}{5pt}		% distance between paragraphs

	Em vez de bloquear de tempo a tempo, uma outra solução para o Master seria controlar com primitivas de sincronização o número de clientes activos e sinalizar o Master quando este valor descesse novamente abaixo do valor máximo estipulado.
	Contudo, esta é uma situação de corrida que não nos convém corrigir.
	O facto de que o Master pode rejeitar uns poucos clientes por alguns segundos porque se baseou numa contagem desactualizada é irrelevante num cenário de várias centenas senão milhares de utilizadores.
	Assim o enquanto que o servidor ainda executa por vezes apenas para se bloquear novamente acaba por ser uma boa situação em relação a bloquear todas as \emph{threads} para ler um inteiro, e adicionar testes ao número de clientes dentro das \emph{workers} pois seriam estas quem teriam a capacidade de gerar o sinal.


	\setlength{\parskip}{20pt}		% distance between paragraphs

	Por fim, o \emph{pool} \emph{Manager} tem a função de capataz responsável por gerir o esforço computacional associado ao atendimento de um dado número de clientes.
	Este dá origem a uma \emph{pool} de \emph{threads} que mantém de forma dinâmica.
	O número exacto de \emph{threads} que são mantidas em execução é actualizado segundo uma taxa definida por \verb|pool_REFRESH_RATE| numa função directa do conjunto de valores:

	\setlength{\parskip}{5pt}		% distance between paragraphs

	\begin{itemize}
		\setlength{\parskip}{-3pt}
		\item{\verb|MIN_WORKERS|}
		\item{\verb|MAX_WORKERS|}
		\item{\verb|MAX_CLIENTS|}
		\item{\verb|pool_HYSTERESIS|}
		\item{\verb|CLIENTS_PER_SLAVE	/* divisão inteira */|}
	\end{itemize}

	Toda a aritmética envolvida no controlo da \emph{pool} é feita sobre inteiros, com resultados inteiros, oferecendo portanto alguma sensibilidade aos valores atribuídos a cada uma das constantes.
	Esta opção é rápida e eficaz, mas obriga a algumas considerações.
	Por análise cuidada do código é possível perceber que situações indesejáveis podem acontecer.

	Nomeadamente, o número de \emph{slaves} nunca pode deve ser inferior a um, sob a penalidade de matar uma \emph{thread} para lançar outra de seguida.
	Esta situação verifica-se por causa das trancas necessárias à correcta execução das \emph{threads} que impedem que estas sejam interrompidas em dados pontos da execução em benefício do cliente.

	Outro exemplo é para uma relação inteira par de clientes por \emph{thread} (e.g. dois clientes por \emph{thread}) com uma histerese nula.
	Nesta situação o \emph{Manager}, por evitar demasiados testes, vai entrar num modo de ressonância quando tem metade dos clientes máximos, lançando uma \emph{thread} e matando-a de seguida.

	Os valores sugeridos para as constantes que condicionam a \emph{pool} foram atingidos por benchmarking, ainda que limitado a um computador único.
	Certamente poderão ser ajustados para uma dada carga de trabalho por forma a optimizar os recursos da máquina que hospeda o servidor sendo portanto uma \emph{pool} bastante configurável.

	\setlength{\parskip}{20pt}		% distance between paragraphs

	Resumindo o atendimento a um cliente, inicialmente, o gerenciador do \emph{pool} lança um número mínimo de \emph{workers}, que poderia até ser nulo, com as implicações já discutidas.
	A partir daí, cada \emph{worker} fica num loop infinito varrendo sequencialmente uma lista de clientes.
	O facto de ser sequencialmente evita simplesmente que um cliente seja deixado por atender, mas não sendo aqui tão eficiente como esquemas em que uma \emph{thread} fica bloqueada num cliente.
	Isto porque um cliente só por existir vai atrasar o processamento dos outros, mesmo quando não envia nenhum comando.
	
	\setlength{\parskip}{5pt}		% distance between paragraphs

	Para percorrer os vários clientes, as \emph{threads} têm de se sincronizar por uma lista global, obrigando a um acesso em exclusividade, utilizando duas estrutura de dados.
	Uma delas correspondendo à lista em si, e outra a um descritor que permite acesso directo a pontos chave da lista sem ter de a percorrer por completo.
	A vantagem deste descritor surge para listas muito grandes, em que é mais fácil manter a informação contida no descritor do que percorrer a lista toda de cada vez.
	Essa outra opção pesaria ainda mais por se tratar de uma estrutura partilhada de acesso em modo de exclusão mútua.

	As esturturas que permitem a identificação e atendimento dos clientes são:
	
	\begin{lstlisting}
	typedef struct clients_descriptor {
		struct client *first;
		struct client *last;
		struct client *next;
		int count;
	} CLIENTS_DESCRIPTOR;

	typedef struct client {
		int fd;
		char IP[INET_ADDRSTRLEN];
		struct client *next;
		struct client *previous;
		struct stack_descriptor *stack_desc;
	} CLIENT;
	\end{lstlisting}
	
	De notar que o mesmo conceito de descritor é usado no \emph{stack}, mais uma vez com prováveis perdas de performance para stacks pequenos, mas que são ao mesmo tempo mais invariáveis ao seu tamanho.

	Depois de escolhido um cliente da lista, as \emph{threads} são completamente iondependentes entre si e livres de qualquer problema de sincronização.
	Um caso interessante surge quando, havendo mais \emph{workers} do que clientes, há mais do que uma \emph{thread} a processar o mesmo cliente.
	Nesta situação só uma é que completa com êxito, não surgindo problema, ainda assim, o servidor foi desenhado tendo em mente vários clientes por cada \emph{worker}.
	
	Ultrapassado o problema de sincronismo, um dos desafios de ter uma só \emph{thread} capaz de dar resposta a todos os clientes ao mesmo tempo, impica o uso de funções de \emph{IO} assíncronas.
	Em particular precisamos agora de desbloquear as \emph{threads} da função \verb|read()| e conforme tenham lido alguma coisa ou não do \emph{\emph{socket}}, mandá-las avançar para outro cliente na lista.
	Este problema tem uma solução trivial, fazendo uso da função \verb|fcntl()|, mas que levanta outro problema: ver secção \ref{sec:servidor_desenvolvimento}.
	


\clearpage
\subsection{Meios de comunicação\label{sec:servidor_comunicacao}}

	\indent\indent O servidor, por ter um esquema de múltiplos fios de execução, tem obrigatoriamente esquemas de comunicação internos a acrescer aos \emph{sockets}.
	
	Internamente, para a passagem dos \emph{file descriptors} aos \emph{workers}, equacionou-se um mecanismo com um \emph{pipe} comum a todas as \emph{threads}, mas o esquema que pretendíamos exigia a presença de certas estruturas para listar os clientes, como já foi referido.
	Por essa razão, o principal meio de comunicação entre os vários fios de execução são na verdade variáveis globais de acesso em regime de exclusão mútua, conseguida através de \emph{mutexes}, incluindo um \emph{mutex} condicional para parar as \emph{threads} quando não há clientes em espera.
	
	Para um outro fim, o \emph{pool Manager} necessita de comunicar com os \emph{workers} que lança.
	Isto serve o intuito de lhes indicar que devem morrer.
	
	Um sinal é aqui a melhor opção, no entanto, não queremos que um sinal interrompa o tratamento de um cliente.
	Por isso bloqueamos o sinal dentro das \emph{threads} mal estas são lançadas e no fim do tratamento de um cliente verificamos se foi lançado o sinal apropriado, \verb|SIG_INT| neste caso.

\clearpage
\subsection{Desenvolvimento futuro\label{sec:servidor_desenvolvimento}}

	\indent\indent O principal elemento em falta no servidor na sua implementação actual é a espera activa sobre a execução dos clientes.
	
	Devido a tornarmos as operações de \emph{IO} sobre os \emph{sockets} assíncronas, ficamos com uma \emph{pool} de \emph{threads} que nunca pára de executar, num loop de espera activa de que um cliente escreve uma mensagem num \emph{\emph{socket}}.
	
	Paradoxalmente, o nosso servidor não consome mais CPU com maior carga de trabalho -- apesar de o fazer com maior número de clientes.
	Isto é na verdade mau sinal, porque na verdade ele consome sempre o máximo (perto do máximo pelo menos) dos recursos que lhe são disponibilizados.
	A excepção à regra verifica-se quando não há nenhum cliente na fila, caso em que um \emph{mutex} condicional bloqueia as \emph{threads}, independentemente do seu número.
	
	A forma de o evitar e contornar esta falha grave poderia passar por duas estratégias, não mutuamente exclusivas.
	
	A primeira, com recurso a funções da família do \verb|select()|.
	Nesta abordagem as \emph{threads} ficam bloqueadas e, na chegada de uma comunicação, poderiam ser lançadas a percorrer a lista de clientes uma única vez.
	Uma ou mais \emph{thread} iriam completar com sucesso o atendimento dos clientes que tinham comunicado alguma coisa, e todas as outras falhavam, mas, invariavelmente, todas se bloqueavam novamente.
	Esta é a solução mais plausível a curto prazo.
	
	Uma segunda alternativa mais apelativa, mas que podia e devia ser combinada com a anterior, permite ainda maiores ganhos de performance em períodos de pouca actividade.
	Ao invés de controlar o número de \emph{threads} na \emph{pool} estaticamente, poderíamos criar uma variável que dava conta da entropia do sistema.
	Quantas mais \emph{threads} terminassem sem atender a nenhum cliente, mais \emph{threads} eram mortas libertando os recursos associados, e vice versa.
		
	Neste esquema, a \emph{pool} não só seria altamente configurável, mas teria um elemento de auto aprendizagem em que o único factor de ajuste seria um ganho a definir pelo administrador.

	Uma forma possível de implementar esta segunda alternativa era incrementar uma variável afectada de um ganho positivo maior do que um, para evitar operações dispendiosas em vírgula flutuante, e decrementar com um ganho potencialmente distinto.
	A incrementação seria executada, por exemplo, no atendimento bem sucedido a um cliente, e decrementada na detecção de um cliente parado.
	
	A partir daqui, o \emph{pool} \emph{Manager} teria apenas de matar ou criar \emph{threads} para as manter a oscilar com este valor.
	Para evitar efeitos negativos com demasiadas criações de \emph{threads} ou eliminações, bastaria manter o esquema de histerese actual.

	Em qualquer um dos casos a implementação do sinal que manda terminar uma \emph{thread} teria de ser revista, pois se as \emph{threads} não estivessem em execução, não poderíamos esperar que elas terminassem logo no esquema actual.


\clearpage
