\section{Desempenho\label{sec:Desempenho}}

	\indent\indent Conscientes de que o nosso projecto está longe de ser eficiente no uso de recursos, pelos problemas já descritos, não deixamos de esperar um bom desempenho no atendimento aos clientes.
	Aliás, o ponto onde o desempenho é sub-óptimo, é precisamente quando o servidor tem clientes registados, mas nenhum requisita qualquer serviço.
	
	Com isto em mente desenhámos um ficheiro de comandos suficientemente longo para que a sua execução não fosse demasiado rápida, e bombardeámos o servidor com clientes a ler os comandos directamente desse ficheiro.
	
	Os parâmetros de ajuste da \emph{pool}, completamente arbitrários, foram:
	
	\begin{lstlisting}
		#define MIN_WORKERS 1
		#define MAX_WORKERS 500
		#define MAX_CLIENTS 1000
		#define CLIENTS_PER_SLAVE ( MAX_CLIENTS / MAX_WORKERS )
		#define POOL_REFRESH_RATE 1		/* [seconds] */
		#define POOL_HYSTERESIS 2
	\end{lstlisting}

	No directório \verb|/Init| é apresentado o output do comando \verb|M| no servidor a meio da execução.
	Não é apresentado aqui pela sua extenção.
	Ao todo lançámos por volta de 200 clientes, e todos eles terminaram a execução em sucesso.
	
	Testes com \verb|MAX_CLIENTS| inferiores ao número de clientes lançados, proporcionaram resultados igualmente positivos.
	
	De notar que estes testes foram executados, por simplicidade, numa única máquina de teste, a correr simultâneamente tanto o servidor como os clientes.
	Por este motivo, o impacto no desempenho do computador foi na verdade superior ao de condições normais, ainda que largamente imperceptíveis na máquina em questão.
	
	
\clearpage
