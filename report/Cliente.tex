\section{Cliente\label{sec:Cliente}}

\subsection{Chamada\label{sec:cliente_chamada}}

	\indent\indent A invocação do cliente segue os requisitos do enunciado, bastando indicar o endereço do servidor e a porta de escuta.
	A falha em passar dois argumentos corresponde a uma falha directa, mas a correcção destes parâmetros não é feita senão mais tarde quando e se o utilizador decidir iniciar sessão.
	\begin{verbatim}
		yascC.exe <hostname> <port>
	\end{verbatim}
	
	Opcionalmente o utilizador poderá ainda fornecer um ficheiro de texto com comandos a executar cuja sintaxe deverá ser idêntica à usada na linha de comandos.
	Para esse efeito deverá ser dado o directório do programa como parâmetro extra, precedido de \verb|-f|.
	A verificação da extensão de ficheiro de texto não é verificada, pelo que o ficheiro usado não tem de ser necessariamente \verb|.txt|.
	\begin{verbatim}
		-f <filename.txt>
	\end{verbatim}

	Ainda no processo de chamada o utilizador pode configurar duas \emph{flags} adicionais: \verb|-l| e \verb|-g|.
	
	A primeira destina-se a suprimir o \emph{output} da linha de comandos e registá-lo na totalidade num ficheiro \verb|log.txt|, excepto erros críticos que continuam a ser dirigidos para o terminal.
	Esta \emph{flag} não pode ser alterada em qualquer outro ponto da execução do cliente e destina-se apenas a permitir uma melhor análise do \emph{output} do programa, sobretudo em situações em que um ficheiro de comandos é usado.
	
	A segunda \emph{flag} permite apenas que o modo \emph{Debug} seja iniciado no arranque, mas o utilizador pode em qualquer momento alternar entre os dois modos de \emph{output}.
	
	Com a excepção do endereço do servidor e da porta de escuta, nenhum destes parâmetros é obrigatório nem tem ordem fixa.
	
	A execução do cliente num directório que não contenha \verb|/doc| não permite o acesso ao ficheiro de ajuda durante a execução.

\clearpage
\subsection{Interface\label{sec:cliente_interface}}

	\indent\indent A interface com o cliente é feita à custa dos comandos:

	\begin{itemize}
		\setlength{\parskip}{-3pt}
		\item{Controlo de sessão:}\begin{itemize}
			\setlength{\parskip}{-3pt}
			\item{I}
			\item{K}
		\end{itemize}
		\item{Operações em \emph{stack}:}\begin{itemize}
			\setlength{\parskip}{-3pt}
			\item{$+$, $-$, $*$, $/$, $\%$}
		\end{itemize}
		\item{Controlo de \emph{stack}:}\begin{itemize}
			\setlength{\parskip}{-3pt}
			\item{P}
			\item{T}
			\item{R}
		\end{itemize}
		\item{Outros:}\begin{itemize}
			\setlength{\parskip}{-3pt}
			\item{G}
			\item{HELP}
			\item{EXIT}
		\end{itemize}
	\end{itemize}

	Os comandos \verb|I| e \verb|K| permitem iniciar e terminar, respectivamente, a ligação ao servidor.
	Esta operação pode ser feita em qualquer ponto ponto da execução do cliente, e depois de terminada a sessão o cliente pode voltar a iniciá-la sob a penalidade de ficar bloqueado na fila do \verb|listen()|.
	No caso de uma sessão ter sido previamente iniciada, \verb|I| deverá limpar o \emph{stack} com retorno positivo \verb|V0|.
	
	As operações aritméticas, todas elas efectuadas sobre números inteiros de $32$ bits, retornam sempre \verb|V0| excepto em caso de overflow, underflow ou divisão por $0$.

	Para visualizar e controlar o \emph{stack}, tem-se ainda os comandos \verb|P|, \verb|T| e \verb|R|.
	Estes permitem saber o tamanho do \emph{stack}, resultados parciais ou números acabados de introduzir e resultados totais, respectivamente.
	O modo mais simples de reinicializá-lo é, como foi dito, com o comando \verb|I|.
	
	A introdução de valores no \emph{stack} é feita simplesmente introduzindo números inteiros em base decimal ou, em alternativa, em base octal na forma \emph{0\#\#\#} ou na base hexadecimal na forma \emph{0x\#\#\#}.
	
	Por fim, os restantes comandos destinam-se a ligar e desligar o modo \emph{Debug}, a abrir um ficheiro na própria linha de comandos, e a fechar o cliente.
	Para a leitura do ficheiro de ajuda, tem ainda de ser tido em conta o directório de execução, conforme explicado na secção \ref{sec:cliente_chamada}, e vale referir que \verb|EXIT| chama internamente \verb|K| por forma a avisar o servidor de que fechou sessão.
	

	O \emph{output} gerado é sempre precedido de sinalética própria do tipo de mensagem em questão:
	
	\begin{tabbing}			% [padding] \= [padding] sets a tab and \kill ignores the line when printing; \> indents to it; \\ changes line
		$\diamond$ \verb|DEBUG:|xx \= \kill
		$\diamond$ \verb|>>| \> Erro que quebra funcionalidade do cliente.\\
		$\diamond$ \verb|: :| \> \emph{Output} normal que inclui algumas mensagens de erro.\\
		$\diamond$ \verb|DEBUG:| \> \emph{Output} extra do modo \emph{Debug}.
	\end{tabbing}

	Num erro crítico o cliente aborta com a mensagem de erro, mas em qualquer outro caso o cliente deverá apenas informar do erro e continuar a executar.
	
	
	Para maior detalhe relativo à interface com o utilizador, pode ser consultado o manual do cliente.
	
	
\clearpage
\subsection{Estrutura lógica\label{sec:cliente_estrutura}}

	\indent\indent O cliente é constituído por um único fio de execução durante todo o seu tempo de execução.
	Permanece num ciclo quási infinito conforme é mostrado no trecho de código seguinte.
	
	De notar que no código abaixo, \verb|fin| identifica a linha de comandos ou o ficheiro de comandos passado ao programa.
	
	\begin{lstlisting}
	while( fgets(line,MAX_LINE,fin) != NULL ) {
		parse_line(line);
	}
	\end{lstlisting}
	
	Neste ciclo, que resume bem o programa, o cliente limita-se a analisar o \emph{input} e a tratá-lo de forma serializada por ordem de introdução.
	Tal inclui a apresentação de mensagens relevantes que obtenha do servidor.
	
	Este tratamento em série do \emph{input} leva a que numa situação de bloqueio à espera do servidor se reflicta num cliente inutilizado.
		

\clearpage
\subsection{Meios de comunicação\label{sec:cliente_comunicacao}}

	\indent\indent O cliente faz uso de um único socket para comunicação com outro processo, potencialmente remoto: o servidor.
	Usa o protocolo TCP/IPv4 por facilidade de teste na máquina de desenvolvimento, e todas as operações executadas sobre esse socket são síncronas.
	
	Esta abordagem facilita a interface com o utilizador, sendo mais fácil gerir qual a resposta de cada comando individual.
	Uma abordagem assíncrona exigiria um buffer de mensagens trocadas entre cliente e utilizador e um maior esforço na sua apresentação.
	
	Por outro lado, a notação polonesa inversa sobressai sobretudo pela sua fácil interpretação síncrona, pelo que esta implementação adequa-se bem aos requisitos do projecto.
	
	
\clearpage
\subsection{Desenvolvimento futuro\label{sec:cliente_desenvolvimento}}

	\indent\indent O cliente neste porjecto é uma entidade relativamente simples, por definição, mas um conjunto de melhorias podia ser posto em prática para omelhorar.
	
	A melhoria mais óbvia seria implementar assincronismo no envio de mensagens e tratamento de respostas, o que levaria a um problema bastante mais complicado tanto do lado do cliente, como mesmo do servidor.
	
	Este assincronismo nunca esteve nos planos para este projecto, mas era um ponto de relevo se quiséssemos expandir o leque de possibilidades para a interface com o servidor.


\clearpage
